%=======================================================================
\chapter{What is \tuprolog{}}
\label{what-is}
%=======================================================================

\tuprolog{} is an open-source, light-weight Prolog framework for distributed applications and infrastructures, released under the LGPL license, available from \url{http://tuprolog.apice.unibo.it}.

Originally developed in/upon Java, which still remains the main reference platform,
\tuprolog{} is currently available for several platforms/environments:
\begin{itemize}
  \item plain JavaSE;
  \item Eclipse plugin;
  \item Android;
  \item Microsoft .NET.
\end{itemize}
%
While they all share the same core and libraries, the latter features an \emph{ad hoc} library which extends the multi-paradigm approach to virtually any language available on the .NET platform (more on this in Chapter \ref{ch:mpp-in-dotnet}).

Unlike most Prolog programming environments, aimed at providing a very efficient (yet monolithic) stand-alone Prolog system, \tuprolog{} is explicitly designed to be \emph{minimal}, dynamically \emph{configurable}, straightforwardly \emph{integrated} with Java and .NET so as to naturally support multi-paradigm/multi-language programming (MPP), and \emph{easily deployable}.

\textit{Minimality} means that its core contains only the Prolog engine essentials -- roughly speaking, the resolution engine and some related basic mechanisms -- for as little as 155KB: any other feature is implemented in \textit{libraries}.
%
So, each user can customize his/her prolog system to fit his/her own needs, and no more:
this is what we mean by \tuprolog{} \textit{configurability}---the necessary counterpart of minimality.

Libraries provide packages of predicates, functors and operators, and can be loaded and unloaded in a \tuprolog{} engine both statically and dynamically.
%
Several standard libraries are included in the \tuprolog{} distribution, and are loaded by default in the standard \tuprolog{} configuration; however, users can easily develop their own libraries either in several ways -- just pure Prolog, just pure Java\footnote{The .NET version of \tuprolog{} supports other languages available on the .NET platform: more on this topic in Chapter \ref{ch:mpp-in-dotnet}}, or a mix of the two --, as we will discuss in Chapter \ref{ch:mpp-in-java}.

\textit{Multi-paradigm programming} is another key feature of \tuprolog{}.
%
In fact, the \tuprolog{} design was intentionally calibrated from the early stages to support a straightforward, pervasive, multi-language/multi-paradigm integration, so as to enable users to:
\begin{itemize}
  \item using any Java\footnote{For the .NET version: any .NET class, library, object, etc.} class, library, object \emph{directly from the Prolog code}
  (Section \ref{sec:java-library}) with no need of pre-declarations, awkward syntax, etc., with full support of parameter passing from the two worlds, yet leaving the two languages and computational models totally separate so as to preserve \emph{a priori} their own semantics---thus bringing the power of the object-oriented platform (e.g. Java Swing, JDBC, etc) to the Prolog world for free;

  \item using any Prolog engine \emph{directly from the Java/.NET code} as one would
   do with any other Java libraries/.NET assemblies (Section \ref{sec:java-api}), again with full support of parameter passing from the two worlds in a non-intrusive, simple way that does not alter any semantics---thus bringing the power of logic programming into virtually \emph{any} Java/.NET application;

  \item augmenting Prolog by defining new libraries (Section \ref{sec:howto-develop-libraries}) either in Prolog, or in the object-oriented language of the selected platform (again, with a straightforward, easy-to-use approach based on reflection which avoids any pre-declaration, language-to-language mapping, etc), or in a mix of both;

  \item augmenting Java\footnote{This feature is currently available only in the Java version: a suitable extension to the .NET platform is under study.} by defining new Java methods in Prolog (the so-called `P@J' framework---Section \ref{sec:p@j}), which exploits reflection and type inference to provide the user with an easy-to-use way to implement Java methods declaratively.
\end{itemize}

Last but not least, \textit{easy deployability} means that the installation requirements are minimal, and that the installation procedure is in most cases\footnote{Exceptions are the Eclipse plugin and the Android versions, which need to be installed as required by the hosting platforms.} as simple as copying one archive to the desired folder.
%
Coherently, a Java-based installation requires only a suitable Java Virtual Machine, and `installing' is just copying a single JAR file somewhere---for as much as 474KB of disk usage (yes, minimality is not just a claim here).
%
Of course, other components can be added (documentation, extra libraries, sources..), but are not necessary for a standard everyday use.
%
The file size is quite similar for the Android platform -- the single APK archive is 234KB -- although an Android-compliant install is performed due to Android requirements.
%
The install process is also quite the same on the .NET platform, although the files are slightly larger.
%
The Eclipse platform also requires a different procedure, since plugin installation have to conform to the requirements of the Eclipse plugin manager: consequently, an update site was set up, where the \tuprolog{} plugin is available as an Eclipse feature. Due to these constraints, file size increases to 1.5MB.


In order to manage all these platforms in a uniform way, a suitable \emph{version numbering scheme} was recently introduced:
\begin{itemize}
  \item the fist two digits represent the engine version;
  \item the last (third) digit is platform-specific and accounts for version differences which do not impact on the Prolog engine -- that is, on the
      \tuprolog{} behaviour -- but simply on graphical aspects or platform-specific
      issues or bugs.
\end{itemize}
%
So, as long as the first two digits are the same, a \tuprolog{} application is guaranteed to behave identically on any supported platform.


Finally, \tuprolog{} also supports \textit{interoperability} with both Internet standard patterns (such as TCP/IP, RMI, CORBA) and coordination models and languages.
%
The latter aspect, in particular, is currently developed in the context of the \tucson{} coordination infrastructure \cite{tucson-aamas99,respect-scico2001}, which provides logic-based, programmable tuple spaces (called \emph{tuple centres}) as the coordination media for distributed processes and agents.\footnote{An alternative infrastructure, \luce{} \cite{luce-aamas2001}, developed the same approach in a location-unaware fashion: this infrastructure is currently no longer supported.}


